\PassOptionsToPackage{unicode=true}{hyperref} % options for packages loaded elsewhere
\PassOptionsToPackage{hyphens}{url}
%
\documentclass[]{article}
\usepackage{lmodern}
\usepackage{amssymb,amsmath}
\usepackage{ifxetex,ifluatex}
\usepackage{fixltx2e} % provides \textsubscript
\ifnum 0\ifxetex 1\fi\ifluatex 1\fi=0 % if pdftex
  \usepackage[T1]{fontenc}
  \usepackage[utf8]{inputenc}
  \usepackage{textcomp} % provides euro and other symbols
\else % if luatex or xelatex
  \usepackage{unicode-math}
  \defaultfontfeatures{Ligatures=TeX,Scale=MatchLowercase}
\fi
% use upquote if available, for straight quotes in verbatim environments
\IfFileExists{upquote.sty}{\usepackage{upquote}}{}
% use microtype if available
\IfFileExists{microtype.sty}{%
\usepackage[]{microtype}
\UseMicrotypeSet[protrusion]{basicmath} % disable protrusion for tt fonts
}{}
\IfFileExists{parskip.sty}{%
\usepackage{parskip}
}{% else
\setlength{\parindent}{0pt}
\setlength{\parskip}{6pt plus 2pt minus 1pt}
}
\usepackage{hyperref}
\hypersetup{
            pdftitle={Lab 10 - Using R Markdown},
            pdfauthor={Bree Bang-Jensen},
            pdfborder={0 0 0},
            breaklinks=true}
\urlstyle{same}  % don't use monospace font for urls
\usepackage[margin=1in]{geometry}
\usepackage{color}
\usepackage{fancyvrb}
\newcommand{\VerbBar}{|}
\newcommand{\VERB}{\Verb[commandchars=\\\{\}]}
\DefineVerbatimEnvironment{Highlighting}{Verbatim}{commandchars=\\\{\}}
% Add ',fontsize=\small' for more characters per line
\usepackage{framed}
\definecolor{shadecolor}{RGB}{248,248,248}
\newenvironment{Shaded}{\begin{snugshade}}{\end{snugshade}}
\newcommand{\AlertTok}[1]{\textcolor[rgb]{0.94,0.16,0.16}{#1}}
\newcommand{\AnnotationTok}[1]{\textcolor[rgb]{0.56,0.35,0.01}{\textbf{\textit{#1}}}}
\newcommand{\AttributeTok}[1]{\textcolor[rgb]{0.77,0.63,0.00}{#1}}
\newcommand{\BaseNTok}[1]{\textcolor[rgb]{0.00,0.00,0.81}{#1}}
\newcommand{\BuiltInTok}[1]{#1}
\newcommand{\CharTok}[1]{\textcolor[rgb]{0.31,0.60,0.02}{#1}}
\newcommand{\CommentTok}[1]{\textcolor[rgb]{0.56,0.35,0.01}{\textit{#1}}}
\newcommand{\CommentVarTok}[1]{\textcolor[rgb]{0.56,0.35,0.01}{\textbf{\textit{#1}}}}
\newcommand{\ConstantTok}[1]{\textcolor[rgb]{0.00,0.00,0.00}{#1}}
\newcommand{\ControlFlowTok}[1]{\textcolor[rgb]{0.13,0.29,0.53}{\textbf{#1}}}
\newcommand{\DataTypeTok}[1]{\textcolor[rgb]{0.13,0.29,0.53}{#1}}
\newcommand{\DecValTok}[1]{\textcolor[rgb]{0.00,0.00,0.81}{#1}}
\newcommand{\DocumentationTok}[1]{\textcolor[rgb]{0.56,0.35,0.01}{\textbf{\textit{#1}}}}
\newcommand{\ErrorTok}[1]{\textcolor[rgb]{0.64,0.00,0.00}{\textbf{#1}}}
\newcommand{\ExtensionTok}[1]{#1}
\newcommand{\FloatTok}[1]{\textcolor[rgb]{0.00,0.00,0.81}{#1}}
\newcommand{\FunctionTok}[1]{\textcolor[rgb]{0.00,0.00,0.00}{#1}}
\newcommand{\ImportTok}[1]{#1}
\newcommand{\InformationTok}[1]{\textcolor[rgb]{0.56,0.35,0.01}{\textbf{\textit{#1}}}}
\newcommand{\KeywordTok}[1]{\textcolor[rgb]{0.13,0.29,0.53}{\textbf{#1}}}
\newcommand{\NormalTok}[1]{#1}
\newcommand{\OperatorTok}[1]{\textcolor[rgb]{0.81,0.36,0.00}{\textbf{#1}}}
\newcommand{\OtherTok}[1]{\textcolor[rgb]{0.56,0.35,0.01}{#1}}
\newcommand{\PreprocessorTok}[1]{\textcolor[rgb]{0.56,0.35,0.01}{\textit{#1}}}
\newcommand{\RegionMarkerTok}[1]{#1}
\newcommand{\SpecialCharTok}[1]{\textcolor[rgb]{0.00,0.00,0.00}{#1}}
\newcommand{\SpecialStringTok}[1]{\textcolor[rgb]{0.31,0.60,0.02}{#1}}
\newcommand{\StringTok}[1]{\textcolor[rgb]{0.31,0.60,0.02}{#1}}
\newcommand{\VariableTok}[1]{\textcolor[rgb]{0.00,0.00,0.00}{#1}}
\newcommand{\VerbatimStringTok}[1]{\textcolor[rgb]{0.31,0.60,0.02}{#1}}
\newcommand{\WarningTok}[1]{\textcolor[rgb]{0.56,0.35,0.01}{\textbf{\textit{#1}}}}
\usepackage{graphicx,grffile}
\makeatletter
\def\maxwidth{\ifdim\Gin@nat@width>\linewidth\linewidth\else\Gin@nat@width\fi}
\def\maxheight{\ifdim\Gin@nat@height>\textheight\textheight\else\Gin@nat@height\fi}
\makeatother
% Scale images if necessary, so that they will not overflow the page
% margins by default, and it is still possible to overwrite the defaults
% using explicit options in \includegraphics[width, height, ...]{}
\setkeys{Gin}{width=\maxwidth,height=\maxheight,keepaspectratio}
\setlength{\emergencystretch}{3em}  % prevent overfull lines
\providecommand{\tightlist}{%
  \setlength{\itemsep}{0pt}\setlength{\parskip}{0pt}}
\setcounter{secnumdepth}{0}
% Redefines (sub)paragraphs to behave more like sections
\ifx\paragraph\undefined\else
\let\oldparagraph\paragraph
\renewcommand{\paragraph}[1]{\oldparagraph{#1}\mbox{}}
\fi
\ifx\subparagraph\undefined\else
\let\oldsubparagraph\subparagraph
\renewcommand{\subparagraph}[1]{\oldsubparagraph{#1}\mbox{}}
\fi

% set default figure placement to htbp
\makeatletter
\def\fps@figure{htbp}
\makeatother


\title{Lab 10 - Using R Markdown}
\author{Bree Bang-Jensen}
\date{1/8/2019}

\begin{document}
\maketitle

\hypertarget{r-markdown-basics}{%
\subsection{R Markdown Basics}\label{r-markdown-basics}}

We will now be doing all of our labs and homework in R Markdown, rather
than in a basic .R script file. This is also how you will write the code
for your own analyses for your projects. Welcome to the future!

Go ahead and open a new R Markdown file -- it's in the pull down menu
where you've previously been selecting a new R Script file.

The basic point of R Markdown is to discuss code, write code, show
results easily, and do all of this in one place. Let's break that down.

\hypertarget{discuss-your-code}{%
\subsubsection{Discuss Your Code}\label{discuss-your-code}}

This is as simple as typing in a Word document. To structure your
discussion, you can add headers and subheaders (as above),
\emph{italics}, and \textbf{bold}. You can also add block quotes.

\begin{quote}
Here's a thing I'd like to emphasize. Maybe a formula. Or an important
fact.
\end{quote}

We can also add horizontal lines. There are other formatting options,
and the R Markdown cheat sheet is a handy way to keep them all straight
(I've also added this PDF to the lab blog for this week:
(\url{https://www.rstudio.com/wp-content/uploads/2015/02/rmarkdown-cheatsheet.pdf})

\begin{center}\rule{0.5\linewidth}{\linethickness}\end{center}

\hypertarget{do-everything-all-in-one-place}{%
\subsubsection{Do Everything All in One
Place}\label{do-everything-all-in-one-place}}

When you click the \textbf{Knit} button and knit to HTML a document will
be generated (and saved in your working directory) that includes both
the written content, as well as any embedded code chunks and the output
of those code chunks within the document.

But wait: what do you mean \emph{embedded code chunks}?

\begin{center}\rule{0.5\linewidth}{\linethickness}\end{center}

\hypertarget{write-code-print-results}{%
\subsubsection{Write Code \& Print
Results}\label{write-code-print-results}}

You can add chunks of code to your R Markdown document using the
\textbf{Insert} button, and selecting R. This is what a basic piece of
embedded R code looks like:

\begin{Shaded}
\begin{Highlighting}[]
\NormalTok{x <-}\StringTok{ }\DecValTok{10}
\NormalTok{x}
\end{Highlighting}
\end{Shaded}

\begin{verbatim}
## [1] 10
\end{verbatim}

\begin{Shaded}
\begin{Highlighting}[]
\KeywordTok{print}\NormalTok{(}\StringTok{"sup"}\NormalTok{)}
\end{Highlighting}
\end{Shaded}

\begin{verbatim}
## [1] "sup"
\end{verbatim}

Your code should all go on lines between the triple accent marks.

There are some important options available to you for each code chunk:

\begin{itemize}
\tightlist
\item
  \textbf{eval} (default is TRUE): Whether to evaluate the code and
  include its results
\item
  \textbf{echo} (default is TRUE): Whether to display code along with
  its results
\item
  \textbf{warning} (default is TRUE): Whether to display warnings
\item
  \textbf{error} (default is FALSE): Whether to display errors
\item
  \textbf{message} (default is TRUE): Whether to display messages
\item
  \textbf{tidy} (default is FALSE): Whether to reformat code in a tidy
  way when displaying it
\item
  \textbf{include}(default is TRUE): Whether to include all output
  (code, results, messages, etc.)
\end{itemize}

Here's a demonstration of where these options go in your code chunk: the
header.

\begin{Shaded}
\begin{Highlighting}[]
\NormalTok{y <-}\StringTok{ }\DecValTok{20}
\NormalTok{y}
\end{Highlighting}
\end{Shaded}

\begin{verbatim}
## [1] 20
\end{verbatim}

This will print the results in the knitted R Markdown document -- in
this case, the value of y -- without printing the code that generated
the results.

In general, you do not want to display errors, warnings, or messages in
your R Markdown document, so you should set those options to FALSE as
needed.

\hypertarget{how-not-to-use-r-markdown}{%
\subsection{How Not to Use R Markdown}\label{how-not-to-use-r-markdown}}

Let's look at an embedded code chunk using a variant of code from an old
homework. Here's a good example of how to make a messy and sad R
Markdown homework.

\begin{center}\rule{0.5\linewidth}{\linethickness}\end{center}

So this is problem 4:

\begin{Shaded}
\begin{Highlighting}[]
\KeywordTok{library}\NormalTok{(}\StringTok{"dplyr"}\NormalTok{)}
\end{Highlighting}
\end{Shaded}

\begin{verbatim}
## 
## Attaching package: 'dplyr'
\end{verbatim}

\begin{verbatim}
## The following objects are masked from 'package:stats':
## 
##     filter, lag
\end{verbatim}

\begin{verbatim}
## The following objects are masked from 'package:base':
## 
##     intersect, setdiff, setequal, union
\end{verbatim}

\begin{Shaded}
\begin{Highlighting}[]
\KeywordTok{library}\NormalTok{(}\StringTok{"gapminder"}\NormalTok{)}
\KeywordTok{library}\NormalTok{(}\StringTok{"ggplot2"}\NormalTok{)}

\NormalTok{gdp_table <-}\StringTok{ }\NormalTok{gapminder }\OperatorTok
\StringTok{  }\KeywordTok{group_by}\NormalTok{(year, continent) }\OperatorTok
\StringTok{  }\KeywordTok{summarize}\NormalTok{(}\DataTypeTok{av_gdp_per_cap =} \KeywordTok{mean}\NormalTok{(gdpPercap))}

\NormalTok{gdp_table}
\end{Highlighting}
\end{Shaded}

\begin{verbatim}
## # A tibble: 60 x 3
## # Groups:   year [12]
##     year continent av_gdp_per_cap
##    <int> <fct>              <dbl>
##  1  1952 Africa             1253.
##  2  1952 Americas           4079.
##  3  1952 Asia               5195.
##  4  1952 Europe             5661.
##  5  1952 Oceania           10298.
##  6  1957 Africa             1385.
##  7  1957 Americas           4616.
##  8  1957 Asia               5788.
##  9  1957 Europe             6963.
## 10  1957 Oceania           11599.
## # ... with 50 more rows
\end{verbatim}

\begin{Shaded}
\begin{Highlighting}[]
\KeywordTok{ggplot}\NormalTok{(gdp_table, }\KeywordTok{aes}\NormalTok{(}\DataTypeTok{x =}\NormalTok{ year, }\DataTypeTok{y =}\NormalTok{ av_gdp_per_cap, }\DataTypeTok{color =}\NormalTok{ continent)) }\OperatorTok{+}\StringTok{ }\KeywordTok{geom_line}\NormalTok{()}
\end{Highlighting}
\end{Shaded}

\includegraphics{Lab-10-Using-R-Markdown_files/figure-latex/unnamed-chunk-4-1.pdf}

\begin{center}\rule{0.5\linewidth}{\linethickness}\end{center}

\hypertarget{a-better-way}{%
\subsection{A Better Way}\label{a-better-way}}

\begin{center}\rule{0.5\linewidth}{\linethickness}\end{center}

\hypertarget{problem-4}{%
\subsubsection{Problem 4}\label{problem-4}}

First, I will load the required packages: dplyr, ggplot2, and the
gapminder data set.

\begin{Shaded}
\begin{Highlighting}[]
\KeywordTok{library}\NormalTok{(}\StringTok{"dplyr"}\NormalTok{)}
\KeywordTok{library}\NormalTok{(}\StringTok{"gapminder"}\NormalTok{)}
\KeywordTok{library}\NormalTok{(}\StringTok{"ggplot2"}\NormalTok{)}
\end{Highlighting}
\end{Shaded}

Then, I will group the gapminder data by year, and then by continent,
and then use the summarize function on each ``year-continent'' group in
order to calculate the mean.

\begin{Shaded}
\begin{Highlighting}[]
\NormalTok{gdp_table <-}\StringTok{ }\NormalTok{gapminder }\OperatorTok
\StringTok{  }\KeywordTok{group_by}\NormalTok{(year, continent) }\OperatorTok
\StringTok{  }\KeywordTok{summarize}\NormalTok{(}\DataTypeTok{av_gdp_per_cap =} \KeywordTok{mean}\NormalTok{(gdpPercap))}
\end{Highlighting}
\end{Shaded}

Here is the head of the resulting table of average GDP by continent by
year.

\begin{verbatim}
## # A tibble: 6 x 3
## # Groups:   year [2]
##    year continent av_gdp_per_cap
##   <int> <fct>              <dbl>
## 1  1952 Africa             1253.
## 2  1952 Americas           4079.
## 3  1952 Asia               5195.
## 4  1952 Europe             5661.
## 5  1952 Oceania           10298.
## 6  1957 Africa             1385.
\end{verbatim}

This table is not the easiest way to conceptualize the data, though: a
graph showing change over time in average GDP for each continent is much
easier to understand.

\begin{Shaded}
\begin{Highlighting}[]
\KeywordTok{ggplot}\NormalTok{(gdp_table, }\KeywordTok{aes}\NormalTok{(}\DataTypeTok{x =}\NormalTok{ year, }\DataTypeTok{y =}\NormalTok{ av_gdp_per_cap, }\DataTypeTok{color =}\NormalTok{ continent)) }\OperatorTok{+}\StringTok{ }\KeywordTok{geom_line}\NormalTok{()}
\end{Highlighting}
\end{Shaded}

\includegraphics{Lab-10-Using-R-Markdown_files/figure-latex/unnamed-chunk-8-1.pdf}

\hypertarget{couple-additional-notes-courtesy-of-karl-broman}{%
\subsubsection{Couple Additional Notes (courtesy of Karl
Broman)}\label{couple-additional-notes-courtesy-of-karl-broman}}

\textbf{Naming code chunks:} It's usually best to give each code chunk a
name, like simulate\_data and chunk\_name above. The name is optional;
if included, each code chunk needs a distinct name. The advantage of
giving each chunk a name is that it will be easier to understand where
to look for errors, should they occur. Also, any figures that are
created will be given names based on the name of the code chunk that
produced them.

Here's where the name would go:

\begin{Shaded}
\begin{Highlighting}[]
\NormalTok{x}
\end{Highlighting}
\end{Shaded}

\begin{verbatim}
## [1] 10
\end{verbatim}

\textbf{Setting global options:} You may be inclined to use largely the
same set of chunk options throughout a document. But it would be a pain
to retype those options in every chunk. Thus, you want to set some
global chunk options at the top of your document.

It could look something like this:

\begin{Shaded}
\begin{Highlighting}[]
\NormalTok{knitr}\OperatorTok{::}\NormalTok{opts_chunk}\OperatorTok{$}\KeywordTok{set}\NormalTok{(}\DataTypeTok{echo =} \OtherTok{FALSE}\NormalTok{, }\DataTypeTok{warning =} \OtherTok{FALSE}\NormalTok{, }\DataTypeTok{message =} \OtherTok{FALSE}\NormalTok{)}
\end{Highlighting}
\end{Shaded}

Because the global chunk options become the defaults for the rest of the
document, you have to specify a different option (within that chunk) if
you want a particular chunk to have a different behavior.

\textbf{Using In-Line Code}: When discussing the results of a given
calculation in your written commentary, we want to be referencing R
calculations, not simply typing numbers that appeared in the console.

That's the point of using in-line code. You'd write something like
``There are 1704 observations in the data.'' Another example: ``The
estimated correlation between x and y was 0.2273181.''

\end{document}
